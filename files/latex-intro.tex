% Before you read this file in detail, it is probably a good idea to run it
% through LaTeX and read the finished document. That will explain the
% principles and the philosophy of LaTeX. Then when you come back to read this
% file in detail you will learn how to write simple LaTeX documents using this
% one as an example.

% All LaTeX documents start off with a plain text file, usually with a
% .tex  file name  extension. These  are then  processed to  produce a
% formatted output.

% Originally LaTeX produced DeVice Independent (dvi) files, using the
% program latex. Nowadays most users prefer output in Portable
% Document Format (pdf) and use a version to produce such files
% directly, namely pdflatex.

% In our first example, we will take the LaTeX source file, this file,
% latex-intro.tex and produce a formatted version, latex-intro.pdf.


% Run the following:
%         Copy this file to an appropriate directory of your own.
%
%         pdflatex latex-intro.tex
%         pdflatex latex-intro.tex
%         evince latex-intro.pdf
%
% and read the processed document.
% Then read the source again from here onwards.
% =============================================================================


% Welcome back! Now you can see how this document was made.

% You can have comments in LaTeX documents which are simply ignored by LaTeX.
% Comments are started by a percent character (without a preceding backslash),
% and continue to the end of the line. This is great, except when you want to
% have a % sign in your document!  (You can include a percent by `escaping' it
% WITH a backslash before it, like this, \%)

% All structure and formatting commands start with a backslash.
% All LaTeX documents start with the \documentclass command.
% This one says the text is to be formatted in 12 point font, for A4 paper.
% And the kind of document is article. Others kinds include report and book.
% Different kinds of document are GENERALLY formatted in different ways
% according to professional best practice, and have different document
% structures allowed.

\documentclass[12pt,a4paper]{article}

% You can use various packages to add features to LaTeX.

% This one makes the page margins wider than the default ones for A4 paper, to save paper.

\usepackage{a4-mancs}

% This one makes the document use fonts which are optimized for postscript
% printers.

\usepackage{pslatex}

% This one offers features to handle uniform resource locators.

\usepackage{url}

% There are loads more, some offer very advanced features.
% You can also write your own packages, but perhaps not just yet!! :-)

% \title defines the title of the document.
% The \LaTeX command produces the fancy version of the string LaTeX.

\title{Probably your first look at \LaTeX}

% \author sets the name of the author

\author{John Latham}

% All the stuff up to now is called the preamble. When starting new documents
% you typically just copy all that from a previous one (and change the title!).

% The document starts here with the following command.

\begin{document}

% This next command makes the title appear.

\maketitle

% This produces a table of contents.
% Warning: CONTENTS might vary. If you want to have the table of contents at
% the beginning of the document (like, when do you not want to?) you will have
% to run pdflatex twice to make sure it gets the page numbers right.

\tableofcontents

% The \newpage commands skips to a new page. Try it by taking it out of a
% comment.

% \newpage

% We divide the document into sections, each starting with the \section
% command. The name of the section is contained in curly brackets.
% Other document structures include \chapter, \subsection and \subsubsection.

\section{Introduction}

% The main text is just presented free format. Except that blank lines mark
% new paragraphs.

% From time to time you put commands in the text. See if you can figure out
% what the second backslash does in \LaTeX\ (hint -- look for an instance of
% \LaTeX without it.)

Hello, and welcome to what is probably your first \LaTeX\ document.

Most expert Unix users prefer to use \LaTeX\ rather than any other document
processing system, because of its inherent power and flexibility, without the
reliance on the user having type-setting expertise. However, it is not
completely trivial to get started using it. Unless you have used it before,
you probably do not even know how to pronounce it!

% A hyphen which is intended to go between words is produced by two hyphens in
% the source text: this gets you a single hyphen of the right size (according
% to the professionals).

So let's get that bit sorted: \LaTeX\ is pronounced ``lay tek'' -- it has
nothing at all to do with rubber!

\section{The \LaTeX\ philosophy}

% You can change the font with environments such as \textbf{}. This one
% formats the text inside the brackets using bold font.

% \emph{} makes the text inside the brackets be emphasized, using italics.

% \ldots produces an ellipses (i.e., three dots).

The philosophy behind \LaTeX\ is this: you concentrate on the \emph{contents}
of your document, and let it worry about the layout. This is in direct
contrast to the \textbf{what you see is what you get} (WYSIWYG) approach of
document processors such as \textbf{Word}. \LaTeX\ users recognise that they
are not really qualified to make good decisions about layout. Professional
type setters, such as those who defined the \LaTeX\ macros, typically laugh at
the average end-user's attempts to make a document look good using a WYSIWYG
tool. Indeed, they coin the interpretation \textbf{what you \emph{sow} is what
you \emph{gather} \ldots}, or less generously \textbf{what you \emph{get} is
what you \emph{deserve}}.

Apparently, many successful authors still prefer to use a type writer for
their early drafts: they feel IMPAIRED when a word processor distracts them
into making the `finished product' look `good', long before it is finished. If
only they had access to a decent plain text editor!

\section{So how do you use \LaTeX?}

The steps are as follows.

% You can have lists of things using environments like itemize. This starts
% with \begin{itemize} and ends with \end{itemize}. Each item starts with
% \item.

\begin{itemize}

% \texttt{} formats the text in the brackets using a type-writer (fixed width)
% font.

\item You create your document using a text editor, such as \textbf{nedit}.
You concentrate on the contents of your document. Let's assume your document
is stored in the file called \texttt{my-first-latex.tex}.

\item You then augment it with `simple' commands to describe its structure.
Some people write these at the same time as the text, but any author who
really wants to focus on the text first and get that right, can go back and
put the structure commands in later.

\item You then run \LaTeX\ to `compile' your document. This will report any
errors, such as mismatched structures, or commands it cannot find, and so on.
To process your document you run the command: \texttt{pdflatex
my-first-latex.tex}.

\item If your document is processed okay, then your \textbf{pdf} file will be
produced. This is a PDF version of your finished document, and
it will be stored in a file called \texttt{my-first-latex.pdf}.

\item You can view your document using a pdf viewer such as \texttt{evince}, or \texttt{acroread}. You
run the command: \texttt{evince my-first-latex.pdf}.

\item You can print your finished document by printing it from your viewer, or just by using \texttt{lpr}.

\end{itemize}

%If instead of using `itemize' we had used `enumerate', then the items would
%have been numbered. Try changing the document in that way.

\section{More information}

I seriously recommend you take a look a using \LaTeX\ for your documents -- it
is a skill well worth acquiring and not nearly as hard as it looks. Most
people can get started on basic documents after only a few minutes of
learning. For example, the \LaTeX\ `expertise' needed for a typical 3rd year
project report can be learnt in a few days. (Alas, if you leave it until that
point in time, you'll feel like you don't have a few days to spare!)

Gradually you can learn how to do advanced things such as define your own
commands. Over time you can learn how to make very complex documents. For
example, I use \LaTeX\ to make lecture slides in \textbf{pdf}, including the
launching of programs by clicking an icon when viewing slides in
\textbf{acroread}.

If you want to learn more about \LaTeX, start by looking at the source of this
document: it contains examples of most of the features you need for simple
documents, and I have included \LaTeX\ comments to explain them.

%If you want to cite a reference then you use \cite{arg} where arg is the
%`logical' name you have given to the reference.

Then look on the web at \cite{software-url} for your next step. (Don't expect
to get any help from \cite{beano}.)

%The references section is an environment called thebibliography. The
%`sample label' given after the start of the environment helps LaTeX know how
%wide your reference labels will be.

\begin{thebibliography}{99}

%Within the thebibliography environment we create entries, each starting with
%the \bibitem command.

\bibitem{software-url}
% \url comes from the url package. When processed by latex2html (you can guess
% what that program does from its name!) it will cause a hyper link to be
% inserted in the resulting html.
\url{http://www.google.co.uk/search?q=latex+tutorial}

\bibitem{beano}
The Beano Annual, any year.

\end{thebibliography}

% This next command ends the document.
\end{document}

Any text you have after the end of the document is ignored. This can be a
handy place to keep general comments and/or old fragments of the document
which you want to keep.

Now, was that so hard? :-) 
