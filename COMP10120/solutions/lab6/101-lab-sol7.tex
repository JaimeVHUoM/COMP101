\documentclass[12pt]{article}
\usepackage{a4-mancs}
\topmargin -80truept
\textheight 265truemm
\usepackage{pslatex}
\usepackage{verbatim}
\usepackage{alltt}
\usepackage{url}
\usepackage{parskip}
%\setlength{\parindent}{0pt}
%\setlength{\parskip}{.7\baselineskip}

\newcommand{\courseName}{COMP10120}
\newcommand{\exerciseDirectory}{\$HOME/COMP10120/ex6}

\title{COMP10120 first semester labs (101L) \\
       Exercise 7: PHP and HTML forms}
     \author{}
     \date{}

\begin{document}
\maketitle

This is a check-list, rather than a normal model answer. You should
be able to use it to quickly check how much students have done, and
then give them a reasonably fair mark for the work.

Students should have:

\begin{itemize}
\item produced web pages on their project (soba, \verb!project_html!) website
  using PHP and HTML

\item linked their (www2, \verb!public_html!) home page to their new pages

\item edited all their PHP etc. by hand

\item used comments to identify code copied from tutorials etc.

\item a page that uses HTML forms to allow users to input their name and
  see it output on another page as part of a welcome message, . . .

\item then test the name to see if it belongs to one of the members of
  their group, and output a different message if so, . . .

\item then accept an email address as well, and use filters to check the
  format of the name and address to avoid possible problems.

\item updated their implementation page with details of software and
  information sources used
\end{itemize}

2 marks for each, total = 16

 0 = not done, trivial attempt (e.g. essentially blank page)

 1 = partial, poor attempt (e.g. not very much on web page)

 2 = essentially ok, good attempt
\end{document}

