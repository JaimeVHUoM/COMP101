\documentclass[12pt]{article}
\usepackage{a4-mancs}
\topmargin -80truept
\textheight 265truemm
\usepackage{pslatex}
\usepackage{verbatim}
\usepackage{alltt}
\usepackage{url}
\usepackage{parskip}
%\setlength{\parindent}{0pt}
%\setlength{\parskip}{.7\baselineskip}

\newcommand{\courseName}{COMP10120}
\newcommand{\exerciseDirectory}{\$HOME/COMP10120/ex6}

\title{COMP10120 first semester labs (101L) \\
       Exercise 9: A simple group web-site, Subversion}
     \author{}
     \date{}

\begin{document}
\maketitle

This is a check-list, rather than a normal model answer. You should
be able to use it to quickly check how much students have done, and
then give them a reasonably fair mark for the work.

The mark for this exercise will be awarded to the group as a whole,
and the mark-sheet should be ordered accordingly.
Please record attendance individually, as normal.

As there will be fewer sites to mark (1 group site instead of e.g. 6
individual sites) please talk to them about:

\begin{itemize}
\item how they spread the work between them
\item whether individuals explained to the rest of the group what they
  done/learnt
\item the quality of what they have done and how they might improve it
\item etc.
\end{itemize}

Each group should have:

\begin{itemize}
\item a group website, using HTML and PHP, with at least the functionality
  expected for the previous lab

\item used CSS

\item a group implementation page with details of software and information
  sources used

\item a group MySQL table to hold relevant information such as name and address

\item a group SVN repository containing all their code, and more than 1
  version of some parts of it

\item edited all their code by hand

\item used comments to identify code authors (within the group, and external)

\item added functionality to link to group-members home pages when they
  log in

\item added functionality to allow users to change the URL for their home
  page

\item added functionality to identify individuals ("sessions") as they log
  in, use the site and log out, avoiding common security problems
\end{itemize}


2 marks for each, total = 20

 0 = not done, trivial attempt (e.g. essentially blank page)

 1 = partial, poor attempt (e.g. not very much on web page)

 2 = essentially ok, good attempt
\end{document}
