\chapter{Customising your Raspberry Pi}

Connect up your Pi as you did last time, and log in as before. Today we're going to explore some of the features of UNIX in a bit more depth. We'll explore some of the more advanced features of the command line and various useful tools that will help you understand how a typical UNIX system is organised, and we'll use the Pi (still in console mode) to browse the Web and read your email. We'll set up a simple web server, and end up by switching into a graphical desktop environment.

We'll start by setting up your Pi so that you can access your University email account. To do this, we're going to need to install an email client (the one we're going to use is called Mutt, though there are many other similar ones available), and configuring it so that it can communicate with the University's email server. 


First let's check that \fname{mutt} isn't already installed on your system (unless you've done this sneakily yourself, it won't be since it's not one of the Pi's default tools). Typing \fname{mutt} on the command line should respond with

\begin{ttoutenv}
-bash: mutt: command not found
\end{ttoutenv}

So we need to install it oursevles. Type:

\begin{ttoutenv}
$ sudo apt-get install mutt
\end{ttoutenv}

and wait for the installation to complete. Let's confirm that a new tool has in fact been installed by typing 

\begin{ttoutenv}
$ which mutt
\end{ttoutenv}

This should respond with \fname{/usr/bin/mutt}, telling us that the \fname{mutt} command has been put in the \fname{/usr/bin} directory on our system. Remember in the last sessions we looked at the contents of the \fname{/bin} directory that contained essential low-level commands such as \ttout{ls}? Well \fname{/usr/bin} contains commands that aren't quite as essential, but have been installed for the users' benefit (i.e. the system would boot/work without them, but it just wouldn't be very useful!)

List the contents of \fname{/usr/bin} by typing
\begin{ttoutenv}
$ ls /usr/bin
\end{ttoutenv}

and notice that here we're using \ttout{ls} to look at the contents of a directory other than the one we're currently in by passing the directory name as a parameter. A whole load of things should scroll past on the screen; most of them won't mean anything to you right now, but don't worry, we'll look at some of the important ones soon enough. Now that's a lot of stuff to look through, and depending on the size of your screen the command we're looking for may have scrolled off the top. So let's try to narrow our results down a bit. Type 
\begin{ttoutenv}
$ ls /usr/bin/m*
\end{ttoutenv}

and you should be given a much smaller list of things from the \fname{/usr/bin} directory; only those starting with the letter m. The asterisk symbol is interpreted as being a `wildcard' that stands for `anything of any length, including length zero', so the command you've just typed means `list the contents of the \fname{/usr/bin} directory, showing only files that start with the letter m and then are followed by zero or more other characters' (notice that the \ttout{man} command that you used in the last session is there amongst the results). 

You could narrow this down even further by typing \ttout{ls /usr/bin/mu*}, in which case you'll only get files from \fname{/usr/bin} that start with the letters mu. Note that if you leave off the asterisk from your command, you'll be asking for files that are called \texttt{exactly} m or mu, and there aren't any of those on the Pi by default. 

So far we've been getting you to do a fair amount of typing, and now we have to admit that you've been typing a lot more than you actually need to (it's good practise though, so we're not feeling too guilty at this stage). The default Linux command line has a feature similar to autocomplete that you'll have seen on web forms and in graphical tools, that saves you typing full commands by suggesting possible alternatives. 

Type \ttout{ls /} but don't hit Enter, and instead press the Tab key twice. You'll be shown a list of sensible things that could follow what you've typed---in this case it's the list of directories that are in the system's root. Now type the letter u (so that the line you've typed so far should read \ttout{ls /u}) and hit Tab once. This time your command will be expanded automatically to \ttout{ls /usr/} since that's the only possible option. Press Tab twice now, and you'll get shown the contents of \fname{/usr/}. Type b, and press Tab to expand the command to \fname{/usr/bin/}, and then press Enter to execute the command.

The `autocomplete' function you're using here is more commonly called `tab complete' in by UNIX users. If you press Tab once and there's only on possible option that would autocomplete what you've typed so far, then that option gets selected; if there are multiple possible things that could complete your command, then pressing Tab a second time shows you those, giving you the option to type another character or two to narrow down the list. Learning to use this will save you a lot of typing, because not only does it reduce the number of characters you type, it also helps you browse the options/files at the same time. 

Here are some other very useful command line tricks:

\begin{itemize}
\item You can use the up and down cursor keys to cycle back and forth through the list of commands you've typed previously.
\item The left and right cursors do what you expect, and move the insertion point back and forth. Pressing \ctrl{a} will move you to the start of the line, and \ctrl{e} to the end of the line (much faster than moving backwards and forwards character-by-character). 
\item \ctrl{c} aborts the current line, so if you've typed a line of gibberish, don't waste time deleting it one character at at time, just \ctrl{c} it!
\item Typing \ttout{history} lists all the commands you've typed in the past, useful if you've forgotten something.
\item Pressing \ctrl{r} allows you to retrieve a command from your history by typing part of the line (e.g. if you searched for `whi' now, it'll probably find the `which mutt' line you typed a while back). 
\item Pressing \ctrl{t} swaps the two characters before your cursor around. What, really? Yes: you'll be surprised how often you type characters in the wrong order! 
\end{itemize}

Back to configuring your email client! Before we use mutt, we need to point it at the incoming and outgoing email servers, and we'll do this by creating a configuration file.

We've created a template file for you to get going with. Make sure you are in your home directory, then use the \ttout{curl} command as in the last lab session to fetch the template from \url{http://pod.cs.man.ac.uk/COMP101/mutt-template}. Remember, you're going to need to use a switch parameter to tell \ttout{curl} what it should call the file it's fetched (call it anything you like, but \fname{mutt-template} is a perfectly good name!) Let's look at the file to see what's in it. Type

\begin{ttoutenv}
$ more ./mutt-template
\end{ttoutenv}

and you should see the following written to the screen:
\begin{ttoutenv}
BLAH
\end{ttoutenv}

The \ttout{more} command is used to display textual content from files and other sources. The reason that the \ttout{more} command is called more will become clear later. 

\begin{note}
  Should they be using 'more' or 'less'
\end{note}

Don't worry too much about the details of this file for now. If you're already familiar with how IMAP and SMTP work together to provide your email service, then you'll be able to see how what the contents of this template mean; if you're not, don't worry, it'll all be explained in detail in the Fundamentals of Distributed Computing course in the second semester. For now, we just need to edit that file to contain your details rather than the fake ones in the template you've just downloaded. But lets play it safe: rather than editing the actual file you downloaded, just in case you make a mistake, let's first make a copy of the file in your home directory. Enter

\begin{ttoutenv}
$ cp mutt-template my-mutt-template
\end{ttoutenv}

Did you type all of that? If so, you've wasted several precious keypresses! You could have typed \ttout{cp mu}, and then pressed Tab to expand it to \ttout{cp mutt-template}, and then added on the \ttout{my-mutt-
template} bit yourself. It's a good habit to get into and will save you a lot of time over the next few years! 

The basic form of the \ttout{cp} command takes two parameters, the first being the file you want to copy, and the second being the name of the file that will be created. Confirm that there is indeed a new file in your home directory using \ttout{ls}, and that its contents are what you expect using \ttout{more} (how would you find out what else the \ttout{cp} command could do?). 

To modify the file, you'll need to use a text editor. Type 
\begin{ttoutenv}
$ nano my-mutt-template
\end{ttoutenv}

to invoke the \ttout{nano} editor, and use it to change the text in square brackets the the correct values for you. Although fairly basic, the nano editor has all the features you'll need to make these changes, and helpfully shows you the various keyboard shortcuts to do particular things such as saving and quitting at the bottom of the screen (the caret symbol (\textasciicircum) is shorthand for `ctrl', so \textasciicircum X means '\ctrl{X}').

\begin{itemize}
\item $[$USERNAME$]$ should be replaced with your university user name (which will be 8 characters, and will start with an m). Note you'll need to replace this on the lines with \ttout{imap\_user} and \ttout{smtp\_url}.
\item $[$FROM$]$ should be DUNNO!
\item $[$REALNAME$]$ is just your real name, in whatever way you want it to appear in outgoing emails.
\end{itemize}

When you've made the changes, write the file to your filestore and quit back to the command line. Then use \ttout{more} to confirm that the file now looks exactly as you want it to. 

Now, \ttout{mutt} expects the file containing its configuration information to have a particular name, and that's not \ttout{my-mutt-template}, so we'll need to do something about that. The UNIX \ttout{mv} command is used to rename files or directories (it's short for `move'), so use that to change the name of the file to \fname{.muttrc} by typing

\begin{ttoutenv}
$ mv my-mutt-template .muttrc
\end{ttoutenv}

Rather like \ttout{cp}, \ttout{mv} takes two parameters; but this time instead of making a copy of the file, \ttout{mv} just changes the name of the file given as the first parameter to that of the second. 

Type \ttout{ls} to confirm that the file name has changed as you'd expect. 

Oh. But it's gone! Actually, no, it's still there, but it's just hidden! There's a UNIX convention that files that start with a full-stop symbol don't appear when you type \ttout{ls} in its basic form, because these are normally configuration files that you don't need to see on a day to day basis (the `rc' part of the '.muttrc' name stands for 'resource configuration', another UNIX convention). So to see these files you'll need to add an extra switch parameter to \ttout{ls}. Use the \ttout{man} command to find out what this switch is, and then use the switch to confirm that the \fname{.muttrc} file does indeed exist. 

Using this switch on \ttout{ls} will reveal several other so-called `dotfiles' that have been lurking in your home directory all along. Use \ttout{more} to look at the contents of the one called \fname{.bash\_history} and it should become obvious how the \ttout{history} command, and the `reverse search' function you used earlier work!

If you're confident that you now have a file called \fname{.muttrc} containing the correct configuration, you can now type \ttout{mutt} to start the program. 

It should be reasonably clear how you use \ttout{mutt} to send and receive email; if you get stuck there are plenty of online tutorials to help you out. Send yourself a test email to make sure that everything is working, and when you're confident you've mastered the basics of sending and reading using this tool, drop back to the command line. One thing you should note is that \ttout{mutt} doesn't have its own editor for writing email, so will use \ttout{nano} to compose emails unless you change this to something else in the \fname{.muttrc} file. 

\section{Browsing the Web}

Although you will have experienced The Web so far as a highly graphical system, the technology that underpins it is for the most part text-based, and it is (just about!) possible to browse web pages using a console-mode application. It might seem like an odd thing to do, but there's an important point to be made here, so bear with us and install the \ttout{lynx} package using \ttout{apt-get} (remembering you'll need also to use \ttout{sudo} to get root privileges). 

Once the package is installed, try browsing the School's web pages using \ttout{lynx} by typing

\begin{ttoutenv}
$ lynx http://www.cs.manchester.ac.uk
\end{ttoutenv}

Rather like \ttout{mutt}, the \ttout{lynx} program has just about enough on-screen help for you to be able to browse around a little without any additional instructions from us.

You may find that when you follow some links, nothing very much appears to have happened; but scroll further down the page and you'll see the content that you're looking for. 

You'll probably find using \ttout{lynx} an unsatisfying experience: tolerable, and probably okay in an emergency, but not how you'd ideally like to browse the web. And you might be wondering why we've even bothered to get you to try `seeing' the web through a text-only interface. Apart from the absence of images and videos etc., the main difference between using something like \ttout{lynx} and a regular browser such as Firefox, Safari or Internet Explorer, is that you'll notice that web pages have been made into much more linear affairs than when they are rendered in a graphical environment. While you might expect to see the navigation links neatly arranged on the left or top of the page with the main content prominently displayed in the centre, seen through a purely textual interface it's all one big stream of stuff, and its very hard to distinguish between the navigation links and the main content. 

Now consider what the web `looks' like if you are visually impaired or blind, and use a screen-reader (a voice-synthesiser program that vocalises the text that's on-screen) to interact with your computer. Whereas a sighted person can easily cope with a two-dimensional layout that allows you to be aware of multiple things at the same time (i.e. you can be reading the main content of the page, but conscious of the fact that there's a navigation bar on the left for when you need it), if instead you are listening to a voice reading the contents of the page out to you, it's only possible to be hearing one thing at a time. And what's more, you have to remember what has been read out in the past in order to make sense of what you are hearing now; you can't just `flick back' a paragraph or two by moving your eyes, instead you have to instruct the screen reader to backtrack and re-read something. So the experience of using the web if you are visually impaired has some things in common to interacting with web-pages using \ttout{lynx}. 

You'll soon be designing your own web-based systems as part of the Group Project in this course unit; making them accessible to visually impaired readers something you should keep in mind. Try using \ttout{lynx} to browse some of your favourite websites, and you'll almost certainly find that the level of `accessibility' on the Web varies considerably!








%\printbibliography

