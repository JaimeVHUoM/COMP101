\documentclass[12pt]{article}
\usepackage{a4-mancs}
\topmargin -80truept
\textheight 265truemm
\usepackage{pslatex}
\usepackage{verbatim}
\usepackage{alltt}
\usepackage{url}
\usepackage{parskip}
%\setlength{\parindent}{0pt}
%\setlength{\parskip}{.7\baselineskip}

\newcommand{\courseName}{COMP10120}
\newcommand{\exerciseDirectory}{\$HOME/COMP10120/ex6}

\title{COMP10120 first semester labs (101L) \\
       Exercise 8: My SQL}
     \author{}
     \date{}

\begin{document}
\maketitle

This is a check-list, rather than a normal model answer. You should
be able to use it to quickly check how much students have done, and
then give them a reasonably fair mark for the work.

Students should have:

\begin{itemize}


\item a MySQL table to hold relevant information such as name and address

\item a simple repeatable mechanism (e.g. script, not all done by hand) for
  initialising their MySQL table for testing

\item updated the web pages on their project (soba, \verb!project_html!) website
  to use MySQL

\item edited all their code by hand

\item PHP/SQL that searches their table for an input name and address

\item PHP/SQL that inserts a new name and address into their table

\item PHP/SQL that outputs a different message depending on whether the name
  is a group member or not, and whether it is new or not

\item validated their input data to avoid possible problems

\item used comments to identify code copied from tutorials etc.

\item updated their implementation page with details of software and
  information sources used

\end{itemize}


2 marks for each, total = 20

 0 = not done, trivial attempt (e.g. essentially blank page)

 1 = partial, poor attempt (e.g. not very much on web page)

 2 = essentially ok, good attempt
\end{document}
