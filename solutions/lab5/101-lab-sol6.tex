\documentclass[12pt]{article}
\usepackage{a4-mancs}
\topmargin -80truept
\textheight 265truemm
\usepackage{pslatex}
\usepackage{verbatim}
\usepackage{alltt}
\usepackage{url}
\usepackage{parskip}
%\setlength{\parindent}{0pt}
%\setlength{\parskip}{.7\baselineskip}

\newcommand{\courseName}{COMP10120}
\newcommand{\exerciseDirectory}{\$HOME/COMP10120/ex6}

\title{COMP10120 first semester labs (101L) \\
       Exercise 6: HTML and CSS}
     \author{}
     \date{}

\begin{document}
\maketitle

This is a check-list, rather than a normal model answer. You should
be able to use it to quickly check how much students have done, and
then give them a reasonably fair mark for the work.

Students should have:

\begin{itemize}
\item produced web pages on their School website using HTML

\item linked their pages together

\item used CSS to enhance the look of their web pages

\item used CSS in a separate style file, rather than included in each page

\item edited all their HTML and CSS by hand

\item produced a home page

\item sensible content (e.g. not embarrassing to them or school)

\item made their page accessible for disabled, elderly, low bandwidth etc.

\item produced correct HTML and checked their pages for HTML errors

\item produced an implementation page, with details of software and
  information sources used e.g. HTML and CSS versions, editor,
  operating system, browser, validator
\end{itemize}


2 marks for each, total = 20

 0 = not done, trivial attempt (e.g. essentially blank page)

 1 = partial, poor attempt (e.g. not very much on web page)

 2 = essentially ok, good attempt
\end{document}
